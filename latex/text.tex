\documentclass[12pt,a4paper]{article}
\usepackage[UTF8]{ctex}
\usepackage{geometry}
\usepackage{graphicx}
\usepackage{amsmath,amssymb}
\usepackage{fancyhdr}
\usepackage{titlesec}
\usepackage{setspace}
\usepackage{indentfirst}
\usepackage{cite}
\usepackage{caption}
\usepackage{hyperref}
\usepackage{enumitem}


% 页面设置
\geometry{a4paper, left=2.5cm, right=2.5cm, top=2.5cm, bottom=2.5cm}
\setlength{\parindent}{2em}
\linespread{1.25}
\pagestyle{fancy}
\fancyhf{} % 清除页眉页脚
\cfoot{\thepage} % 页码居中
\renewcommand{\headrulewidth}{0pt} % 去掉页眉横线

% 标题格式设置
\titleformat{\section}{\centering\zihao{4}\heiti}{\thesection}{1em}{}
\titleformat{\subsection}{\zihao{-4}\heiti}{\thesubsection}{1em}{}
\titleformat{\subsubsection}{\zihao{-4}\heiti}{\thesubsubsection}{1em}{}

% 正文小四宋体
\renewcommand{\normalsize}{\zihao{-4}\songti\linespread{1.25}\selectfont}

% 摘要页为第一页
\begin{document}

% 第一页 摘要页
\thispagestyle{empty}
\begin{center}
    \zihao{3}\heiti康养城市建设(暂定) \\
    \vspace{1em}
    \zihao{4}\heiti选题编号:A \\
    \quad 队伍编号:202503192 \\
\end{center}

\vspace{2em}

\noindent\textbf{摘要:}(填写摘要内容,建议控制在一页以内。)

\vspace{1em}

\noindent\textbf{关键词:}关键词1;关键词2;关键词3

\newpage

\twocolumn[]
% 正文
\section{引言}

随着我国人口老龄化进程加快,康养城市建设正逐步成为推动社会
可持续发展的重要战略方向。在2025年全国两会期间,
多位代表委员聚焦康养议题,提出诸如盘活空置房、
发展数字家庭、推动医养融合等富有前瞻性的建议,
引发社会广泛关注。各地政府亦积极响应,
安徽出台《关于支持康养产业高质量发展的意见》\cite{01},
广德市加快“康养名城”建设步伐,吉林提出打造旅居康养目的地,
种种举措展现出康养城市建设已从政策倡导迈向系统实践。
然而,现实中康养城市建设仍面临诸多挑战:
康养资源在区域间分布不均,智慧养老技术应用滞后,
医养融合程度不足,制约了康养服务体系的整体效能。
因此,科学评估当前康养资源配置状况,构建合理的评价体系,
并设计优化资源配置的策略,成为推进康养城市高质量发展的当务之急。
为此,本文将围绕以下三方面开展研究:

\textbf{一、对城市康养资源分布现状进行分析,揭示资源配置的短板与优化空间;}

\textbf{二、构建康养城市综合评价模型,量化城市康养发展水平;}

\textbf{三、建立康养资源优化配置模型,制定切实可行的实施策略,旨在为地方政府推动康养城市建设提供科学依据与决策支持。}

\section{模型建立}

\subsection{康养指标选取标准}

(1) 相关性:指标应与康养城市建设密切相关,能够反映康养资源的配置状况。

(2) 可获取性:指标数据应易于获取,确保模型的可操作性。

(3) 代表性:指标应具有一定的代表性,能够反映城市整体的康养发展水平。

(4) 可比性:指标应具有一定的可比性,便于不同城市之间进行横向比较。

(5) 时效性:指标数据应具有一定的时效性,能够反映当前的康养资源配置状况。

(6) 可量化性:指标应具有一定的可量化性,便于进行定量分析。

(7) 适用性:指标应适用于不同类型的城市,具有一定的普遍适用性。

\subsection{康养指标}

(1) 生活环境:良好的生活环境有助于减少疾病传播、改善心理状态、提高生活质量,是维持身体健康和延年益寿的重要基础。

(2) 医疗服务:医疗服务能及时预防、诊断和治疗疾病,提升健康管理水平,有效降低死亡率和患病风险,是保障公众健康的关键力量。

(3) 经济水平:经济水平决定了医疗保障、营养摄入和生活质量,高水平经济可提供更完善的健康服务体系,为居民健康提供有力支撑。

(4) 人口结构:人口结构影响健康资源需求与配置,老龄化社会需更多医疗与养老服务,合理的人口结构有助于实现健康服务的可持续发展。

(5) 生活成本:生活成本影响人们获取健康食物、医疗服务与良好居住条件的能力,成本过高可能导致压力增加和健康资源不足,从而危害身心健康。

(6) 教育资源:教育资源提升健康意识与自我管理能力,促使人们养成良好生活习惯,有助于预防疾病、改善心理健康,从根本上促进全民健康水平。

(7) 社区服务:社区服务能提供便捷的医疗、养老、心理支持等服务,增强居民健康管理与应急能力,是提升全民健康水平的重要基础设施。

\subsection{量化康养指标}

(1)人均资源密度
\[
\rho = k_0 \cdot \frac{1}{S \cdot P}
\]
其中,$\rho$为人均资源密度,$k_0$为常数,$S$为城市面积,$P$为城市人口。

\subsection{康养指数公式}

\[A = \sum_{i=1}^{7} K_i \times W_i^{[1]}\]

其中A为康养指数,K为各项指标权重,W为各项指标值

\section{数据分析与结果}
图片插入示例:
\begin{figure}[h]
    \centering
    \includegraphics[width=0.5\textwidth]{example-image} % 替换为你的图片路径
    \caption{示意图}
\end{figure}

\section{结论与讨论}
正文内容。

% 附录
\newpage
\section*{附录}
附录内容,可包含推导、补充图表等。

\onecolumn
% 参考文献
\newpage
\begin{thebibliography}{99}
        \bibitem{01} 王峰.我省出台意见支持康养产业高质量发展[N].合肥日报,2024-11-13(001).
        \bibitem{02} 房红,张旭辉.康养产业:概念界定与理论构建[J].四川轻化工大学学报(社会科学版),2020,35(04):1-20.
        \bibitem{03} 郑自君,袁东升,房鹏,等.攀西地区森林康养指数综合分析[J].气象科技,2021,49(05):815-822.DOI:10.19517/j.1671-6345.20200559.

\end{thebibliography}
    % 添加更多参考文献


\end{document}
