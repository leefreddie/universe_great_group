\documentclass[12pt,a4paper]{article}
\usepackage[UTF8]{ctex}
\usepackage{geometry}
\usepackage{graphicx}
\usepackage{amsmath,amssymb}
\usepackage{fancyhdr}
\usepackage{titlesec}
\usepackage{setspace}
\usepackage{indentfirst}
\usepackage{cite}
\usepackage{caption}
\usepackage{hyperref}
\usepackage{enumitem}
\usepackage{multirow}
\usepackage{longtable}
\usepackage{booktabs}
\usepackage{makecell}
\usepackage{multicol}


% 页面设置
\geometry{a4paper, left=2.5cm, right=2.5cm, top=2.5cm, bottom=2.5cm}
\setlength{\parindent}{2em}
\linespread{1.25}
\pagestyle{fancy}
\fancyhf{} % 清除页眉页脚
\cfoot{\thepage} % 页码居中
\renewcommand{\headrulewidth}{0pt} % 去掉页眉横线

% 标题格式设置
\titleformat{\section}{\centering\zihao{4}\heiti}{\thesection}{1em}{}
\titleformat{\subsection}{\zihao{-4}\heiti}{\thesubsection}{1em}{}
\titleformat{\subsubsection}{\zihao{-4}\heiti}{\thesubsubsection}{1em}{}

% 正文小四宋体
\renewcommand{\normalsize}{\zihao{-4}\songti\linespread{1.25}\selectfont}

% 摘要页为第一页
\begin{document}

% 第一页 摘要页
\thispagestyle{empty}
\begin{center}
    \zihao{3}\heiti康养城市建设(暂定) \\
    \vspace{1em}
    \zihao{4}\heiti选题编号:A \\
    \quad 队伍编号:202503192 \\
\end{center}

\vspace{2em}

\noindent\textbf{摘要:}(填写摘要内容,建议控制在一页以内。)

\vspace{1em}

\noindent\textbf{关键词:}关键词1;关键词2;关键词3

\newpage

% 正文
\section{引言}

随着我国人口老龄化进程加快,康养城市建设正逐步成为推动社会
可持续发展的重要战略方向。在2025年全国两会期间,
多位代表委员聚焦康养议题,提出诸如盘活空置房、
发展数字家庭、推动医养融合等富有前瞻性的建议,
引发社会广泛关注。各地政府亦积极响应,
安徽出台《关于支持康养产业高质量发展的意见》\cite{01},
广德市加快“康养名城”建设步伐,吉林提出打造旅居康养目的地,
种种举措展现出康养城市建设已从政策倡导迈向系统实践。
然而,现实中康养城市建设仍面临诸多挑战:
康养资源在区域间分布不均,智慧养老技术应用滞后,
医养融合程度不足,制约了康养服务体系的整体效能。
因此,科学评估当前康养资源配置状况,构建合理的评价体系,
并设计优化资源配置的策略,成为推进康养城市高质量发展的当务之急。
为此,本文将围绕以下三方面开展研究:

\textbf{一、对城市康养资源分布现状进行分析,揭示资源配置的短板与优化空间;}

\textbf{二、构建康养城市综合评价模型,量化城市康养发展水平;}

\textbf{三、建立康养资源优化配置模型,制定切实可行的实施策略,旨在为地方政府推动康养城市建设提供科学依据与决策支持。}

\begin{multicols}{2}

\section{模型建立}

\subsection{康养指标选取标准}

(1) 相关性:指标应与康养城市建设密切相关,能够反映康养资源的配置状况。

(2) 可获取性:指标数据应易于获取,确保模型的可操作性。

(3) 代表性:指标应具有一定的代表性,能够反映城市整体的康养发展水平。

(4) 可比性:指标应具有一定的可比性,便于不同城市之间进行横向比较。

(5) 时效性:指标数据应具有一定的时效性,能够反映当前的康养资源配置状况。

(6) 可量化性:指标应具有一定的可量化性,便于进行定量分析。

(7) 适用性:指标应适用于不同类型的城市,具有一定的普遍适用性。

\subsection{康养指标}

(1) 生活环境:良好的生活环境有助于减少疾病传播、改善心理状态、提高生活质量,是维持身体健康的重要基础。

(2) 医疗服务:医疗服务能及时预防、诊断和治疗疾病,提升健康管理水平,有效降低死亡率和患病风险,是保障公众健康的关键力量。

(3) 经济水平:经济水平决定了医疗保障、营养摄入和生活质量,高水平经济可提供更完善的健康服务体系,为居民健康提供有力支撑。

(4) 人口结构:人口结构影响健康资源需求与配置,老龄化社会需更多医疗与养老服务,合理的人口结构有助于实现健康服务的可持续发展。

(5) 生活成本:生活成本影响人们获取健康食物、医疗服务与良好居住条件的能力,成本过高可能导致压力增加和健康资源不足,从而危害身心健康。

(6) 教育资源:教育资源提升健康意识与自我管理能力,促使人们养成良好生活习惯,有助于预防疾病、改善心理健康,从根本上促进全民健康水平。

(7) 社区服务:社区服务能提供便捷的医疗、养老、心理支持等服务,增强居民健康管理与应急能力,是提升全民健康水平的重要基础设施。

\subsection{量化康养指标}

\subsubsection{资源密度}
\[
\rho = \frac{R}{S}
\]

其中,$\rho$为资源密度,$R$为资源总量,$S$为区域面积。

\subsubsection{人口密度}
\[
\delta = \frac{P}{S}
\]

其中,$\delta$为人口密度,$P$为区域人口总数,$S$为区域面积。

\subsubsection{资源配置密度/人均资源数量}
\[
E = \frac{\rho}{\delta}\cdot 10^5=\frac{R}{P}\cdot 10^5
\]

其中,$E$为资源配置均衡度,$10^5$校准数量级,$R$为资源总量,$P$为区域人口总数。\textbf{当$E$越大,说明人均资源越丰富。}

\subsubsection{康养资源倾斜度}

\[
W=\frac{R}{C}
\]

其中,$W$为康养资源倾斜度,$R$为资源总量,$C$为年人均收入。

\subsubsection{资源分配系数}

\[
G=\frac{R}{P\cdot C}
\]
其中,$G$为资源分配系数,$R$为资源总量,$P$为区域人口总数,$C$为年人均收入。

\subsubsection{资源配置合理性}
\[
\sigma = \sqrt{\frac{1}{n} \sum_{i=1}^{n}(G_i - \mu)^2}
\]
其中,$\mu = \frac{1}{n} \sum_{i=1}^{n}G_i$,是全体$G$的均值,$E_i$为第i个城市的资源配置密度,n为数据数量(个数),$\sigma$为全体E的方差,代表资源配置合理性。\textbf{当$\sigma$越小,说明资源配置越合理。}

\subsection{康养指数}

\subsubsection{康养指数公式}

\[
A = a_1E + a_2W + a_3G - a_4\sigma
\]

其中A为康养指数,K为各项指标权重,E为资源配置均衡度,$\sigma$为资源配置合理性,$W$为康养资源倾斜度。

\subsubsection{权重计算}

我们将采用熵值法\cite{02}计算权重。
熵值法是一种客观赋权方法,能够有效消除主观因素对权重的影响。其基本步骤如下:

(1) 数据标准化:将原始数据进行标准化处理,使其符合正态分布。

(2) 计算熵值:根据标准化后的数据,计算各指标的熵值。

(3) 计算权重:根据熵值计算各指标的权重。

(4) 归一化处理:将权重进行归一化处理,使其和为1。

\section{数据计算}

接下来以上海市为例,计算资源配置均衡度,并分析其资源分布合理性。
\subsection{资源配置合理性计算}



\section{数据分析与结果}
图片插入示例:
\begin{figure}[h]
    \centering
    \includegraphics[width=0.5\textwidth]{example-image} % 替换为你的图片路径
    \caption{示意图}
\end{figure}

\section{结论与讨论}

\end{multicols}
% 附录
\newpage
\section*{附录}

\begin{table}[h]
  \centering
  \caption{康养城市综合评价指标数据列表}
  \begin{tabular}{c|c|c|c|ccccc}
    \toprule[2pt]
    \multirow{2}{*}{城市} & \multirow{2}{*}{人口数量/万} & \multirow{2}{*}{城市面积/$km^2$}& \multirow{2}{*}{年平均收入/元}& \multicolumn{5}{c}{公共设施数量} \\
\cline{5-9}
& & & & 社区 & 医院 & 养老院 & 公园 & 院校 \\
    \midrule[1pt]
    常州 & 538.6 & 4385.00 &59514& 96 & 55 & 89 & 20 & 10 \\
    广州 & 1897.8 & 7434.40 &76849& 90 & 73 & 118 & 69 & 35 \\
    海口 & 287.3 & 2296.82 &38361& 79 & 63 & 122 & 55 & 12 \\
    南京 & 954.7 & 6587.04 &69039& 100 & 85 & 68 & 64 & 80\\
    上海 & 2480.26 & 6340.50 &84034& 58 & 176 & 103 & 112 & 72 \\
    \bottomrule[2pt]
  \end{tabular}

  \vspace{0.5em}
  {\footnotesize 注:生态指数数据来自中国生态环境部,公共设施数量信息来自百度地图。(下表同)}
\end{table}


\begin{longtable}{c|c|c|c|ccccc}
\caption{上海-康养城市综合评价指标数据列表} \\
\toprule[2pt]
\multirow{2}{*}{区域} & \multirow{2}{*}{人口数量/万} & \multirow{2}{*}{地区面积/$km^2$}& \multirow{2}{*}{年人均收入\cite{05}/元}& \multicolumn{5}{c}{公共设施数量} \\
\cline{5-9}
& & & & 社区 & 医院 & 养老院 & 公园 & 院校 \\
\midrule[1pt]
\endfirsthead

\multicolumn{9}{c}{续表:上海-康养城市综合评价指标数据列表} \\
\toprule[2pt]
\multirow{2}{*}{区域} & \multirow{2}{*}{人口数量/万} & \multirow{2}{*}{地区面积/$km^2$}& \multirow{2}{*}{年人均收入\cite{05}/元}& \multicolumn{5}{c}{公共设施数量} \\
\cline{5-9}
& & & & 社区 & 医院 & 养老院 & 公园 & 院校 \\
\midrule[1pt]
\endhead

\hline
\multicolumn{9}{r}{表格续下页} \\
\endfoot

\hline
\endlastfoot

黄浦 & 66.2 & 20.46 & 96448 & 1 & 7 & 3 & 6 & 3 \\
徐汇 & 111.3 & 54.93 & 90555 & 9 & 12 & 5 & 5 & 11 \\
长宁 & 69.3 & 38.30 & 92402 & 4 & 5 & 1 & 4 & 5 \\
静安 & 97.6 & 36.88 & 93547 & 2 & 15 & 5 & 6 & 2 \\
普陀 & 124.0 & 54.83 & 88916 & 8 & 4 & 3 & 6 & 2 \\
虹口 & 75.7 & 23.46 & 90959 & 1 & 8 & 4 & 5 & 3 \\
杨浦 & 124.3 & 60.73 & 90529 & 1 & 13 & 5 & 8 & 7 \\
闵行 & 265.3 & 372.56 & 82413 & 3 & 8 & 15 & 9 & 4 \\
宝山 & 223.5 & 365.30 & 79344 & 1 & 5 & 5 & 5 & 5 \\
嘉定 & 183.4 & 463.55 & 67277 & 1 & 9 & 7 & 10 & 2 \\
浦东 & 568.1 & 1210.41 & 84089 & 10 & 36 & 16 & 20 & 10 \\
金山 & 82.3 & 613.00 & 53817 & 1 & 5 & 5 & 2 & 1 \\
松江 & 191.0 & 604.64 & 66452 & 5 & 4 & 6 & 9 & 3 \\
青浦 & 127.1 & 668.54 & 59944 & 3 & 5 & 1 & 2 & 0 \\
奉贤 & 114.1 & 733.38 & 55292 & 2 & 2 & 5 & 3 & 4 \\
崇明 & 63.8 & 1413.00 & 48237 & 1 & 1 & 11 & 4 & 4 \\

\bottomrule[2pt]

\end{longtable}


% 参考文献
\newpage
\begin{thebibliography}{99}
        \bibitem{01} 王峰.我省出台意见支持康养产业高质量发展[N].合肥日报,2024-11-13(001).
        \bibitem{02} 王靖,张金锁.综合评价中确定权重向量的几种方法比较[J].河北工业大学学报,2001,(02):52-57.
        \bibitem{03} 房红,张旭辉.康养产业:概念界定与理论构建[J].四川轻化工大学学报(社会科学版),2020,35(04):1-20.
        \bibitem{04} 郑自君,袁东升,房鹏,等.攀西地区森林康养指数综合分析[J].气象科技,2021,49(05):815-822.DOI:10.19517/j.1671-6345.20200559.
        \bibitem{05} 腾讯新闻. 上海和北京的各辖区人均收入排名[EB/OL]. \url{https://news.qq.com/rain/a/20230817A0357X00}, 访问时间:2025年5月22日。

\end{thebibliography}
    % 添加更多参考文献


\end{document}
