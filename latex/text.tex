\documentclass[12pt,a4paper]{article}
\usepackage[UTF8]{ctex}
\usepackage{geometry}
\usepackage{graphicx}
\usepackage{amsmath,amssymb}
\usepackage{fancyhdr}
\usepackage{titlesec}
\usepackage{setspace}
\usepackage{indentfirst}
\usepackage{cite}
\usepackage{caption}
\usepackage{hyperref}

% 页面设置
\geometry{a4paper, left=2.5cm, right=2.5cm, top=2.5cm, bottom=2.5cm}
\setlength{\parindent}{2em}
\linespread{1.25}
\pagestyle{fancy}
\fancyhf{} % 清除页眉页脚
\cfoot{\thepage} % 页码居中
\renewcommand{\headrulewidth}{0pt} % 去掉页眉横线

% 标题格式设置
\titleformat{\section}{\centering\zihao{4}\heiti}{\thesection}{1em}{}
\titleformat{\subsection}{\zihao{-4}\heiti}{\thesubsection}{1em}{}
\titleformat{\subsubsection}{\zihao{-4}\heiti}{\thesubsubsection}{1em}{}

% 正文小四宋体
\renewcommand{\normalsize}{\zihao{-4}\songti\linespread{1.25}\selectfont}

% 摘要页为第一页
\begin{document}

% 第一页 摘要页
\thispagestyle{empty}
\begin{center}
    \zihao{3}\heiti康养城市建设(暂定) \\
    \vspace{1em}
    \zihao{4}\heiti选题编号:A \quad 队伍编号:202503192 \\
\end{center}

\vspace{2em}

\noindent\textbf{摘要:}(填写摘要内容,建议控制在一页以内。)

\vspace{1em}

\noindent\textbf{关键词:}关键词1;关键词2;关键词3

\newpage

% 正文
\section{引言}
正文内容。

\section{模型建立}
数学公式示例:
\[
E = mc^2
\]

\section{数据分析与结果}
图片插入示例:
\begin{figure}[h]
    \centering
    \includegraphics[width=0.5\textwidth]{example-image} % 替换为你的图片路径
    \caption{示意图}
\end{figure}

\section{结论与讨论}
正文内容。

% 附录
\newpage
\section*{附录}
附录内容,可包含推导、补充图表等。

% 参考文献
\newpage
\section*{参考文献}
\begin{enumerate}
    \item 张三,数学建模方法,北京:清华大学出版社,2015年。
    \item 李四,最优化方法综述,系统工程,27(3):45-52,2011年。% chktex 8忽略破折号警告
    \item 王五,某某研究进展,\url{http://example.com},访问时间:2025年5月20日。
\end{enumerate}

\end{document}
