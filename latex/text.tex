\documentclass[12pt, a4paper]{article}

% 中文支持(如不需要可注释掉)
\usepackage[UTF8]{ctex} % 使用 XeLaTeX 编译
% 英文支持
\usepackage[english]{babel}

% 页边距
\usepackage[a4paper, margin=1in]{geometry}

% 数学公式支持
\usepackage{amsmath, amssymb, amsthm}

% 图表支持
\usepackage{graphicx}
\usepackage{caption}
\usepackage{subcaption}

% 代码块支持
\usepackage{listings}
\usepackage{xcolor}

% 超链接支持
\usepackage{hyperref}
\hypersetup{
    colorlinks=true,
    linkcolor=blue,
    citecolor=blue,
    urlcolor=blue
}

% 自定义封面信息
\title{\Huge\bfseries This is Title}
\author{\Large Universal Great Group}
\date{\Large \today}

% 定理环境(可选)
\newtheorem{theorem}{Theorem}[section]
\newtheorem{definition}{Definition}[section]
\newtheorem{lemma}{Lemma}[section]

\begin{document}

% 封面
\maketitle
\thispagestyle{empty} % 去除页眉页脚
\newpage

% 目录
\tableofcontents
\newpage

% 正文部分示例
\section{引言}

这是引言部分的内容。

\section{预备知识}

\subsection{定义与定理}

\begin{definition}
这是一个定义的例子。
\end{definition}

\begin{theorem}
这是一个定理。
\end{theorem}

\begin{proof}
这是证明过程。
\end{proof}

\section{主要结果}

主内容可继续在这里书写。

\section{结论}

总结你的论文工作。

\newpage
\begin{thebibliography}{9}
\bibitem{example} 某某人, \emph{书籍或文章标题}, 出版社, 年份.
\end{thebibliography}

\end{document}
