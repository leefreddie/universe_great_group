\documentclass[12pt,a4paper]{article}
\usepackage[UTF8]{ctex}
\usepackage{geometry}
\usepackage{graphicx}
\usepackage{amsmath,amssymb}
\usepackage{fancyhdr}
\usepackage{titlesec}
\usepackage{setspace}
\usepackage{indentfirst}
\usepackage{cite}
\usepackage{caption}
\usepackage{hyperref}
\usepackage{enumitem}
\usepackage{multirow}
\usepackage{longtable}
\usepackage{booktabs}
\usepackage{makecell}
\usepackage{multicol}
\usepackage{float}
\usepackage{subfigure}
\usepackage{adjustbox}
\usepackage{algorithm}
\usepackage{algpseudocode} % 更好的伪代码排版
\usepackage{amsmath} % 数学公式

% 页面设置
\geometry{a4paper, left=2.5cm, right=2.5cm, top=2.5cm, bottom=2.5cm}
\setlength{\parindent}{2em}
\linespread{1.25}
\pagestyle{fancy}
\fancyhf{} % 清除页眉页脚
\cfoot{\thepage} % 页码居中
\renewcommand{\headrulewidth}{0pt} % 去掉页眉横线

% 标题格式设置
\titleformat{\section}{\centering\zihao{4}\heiti}{\thesection}{1em}{}
\titleformat{\subsection}{\zihao{-4}\heiti}{\thesubsection}{1em}{}
\titleformat{\subsubsection}{\zihao{-4}\heiti}{\thesubsubsection}{1em}{}

% 正文小四宋体
\renewcommand{\normalsize}{\zihao{-4}\songti\linespread{1.25}\selectfont}

% 摘要页为第一页
\begin{document}

% 第一页 摘要页
\thispagestyle{empty}
\begin{center}
    \zihao{3}\heiti基于Delaunnay三角剖分的康养资源分布情况分析 \\
    \vspace{1em}
    \zihao{4}\heiti选题编号:A \\
    \quad 队伍编号:202503192 \\
\end{center}

\vspace{2em}

\noindent\textbf{摘要:}本文围绕康养城市建设问题,针对康养资源分布不均、医养融合深度不够等问题,建立了康养指标模型,并结合上海市16市、常州、南京、海口、广州的医院、养老院分布结合2024中国康养城市100强名单等进行了实证研究。首先,通过计算简化版的康养指标得出康养指标得分越高的城市该城市在名单中的排名越高,推出建立的简化模型可以在一定程度上工作;接着,利用python批量带入数据进一步验证了模型的有效性;最后对常州、广州、南京、海口、上海五个城市进行了Delaunay三角剖分进行空间分布分析。研究结果表明该模型能很直观地反映康养资源的分布情况,对于资源分布调整及建设有很大帮助。本研究对城市规划,GIS相关,资源配置等具有一定的参考价值。

\vspace{1em}

\noindent\textbf{关键词:}Delaunay三角剖分;康养城市;资源配置;空间分布;熵值法

\newpage

\setcounter{page}{1}
% 正文
\twocolumn[]

\section{引言}

随着我国人口老龄化进程加快,康养城市建设正逐步成为推动社会
可持续发展的重要战略方向。在2025年全国两会期间,
多位代表委员聚焦康养议题,提出诸如盘活空置房、
发展数字家庭、推动医养融合等富有前瞻性的建议,
引发社会广泛关注。各地政府亦积极响应,
安徽出台《关于支持康养产业高质量发展的意见》\cite{01},
广德市加快"康养名城"建设步伐,吉林提出打造旅居康养目的地,
种种举措展现出康养城市建设已从政策倡导迈向系统实践。
然而,现实中康养城市建设仍面临诸多挑战:
康养资源在区域间分布不均,智慧养老技术应用滞后,
医养融合程度不足,制约了康养服务体系的整体效能。
因此,科学评估当前康养资源配置状况,构建合理的评价体系,
并设计优化资源配置的策略,成为推进康养城市高质量发展的当务之急。
为此,本文将围绕以下三方面开展研究:

\textbf{一、对城市康养资源分布现状进行分析,揭示资源配置的短板与优化空间;}

\textbf{二、构建康养城市综合评价模型,量化城市康养发展水平;}

\textbf{三、建立康养资源优化配置模型,制定切实可行的实施策略,旨在为地方政府推动康养城市建设提供科学依据与决策支持。}

\section{模型建立}

\subsection{康养指标选取标准}

(1) 相关性:指标应与康养城市建设密切相关,能够反映康养资源的配置状况。

(2) 可获取性:指标数据应易于获取,确保模型的可操作性。

(3) 代表性:指标应具有一定的代表性,能够反映城市整体的康养发展水平。

(4) 可比性:指标应具有一定的可比性,便于不同城市之间进行横向比较。

(5) 时效性:指标数据应具有一定的时效性,能够反映当前的康养资源配置状况。

(6) 可量化性:指标应具有一定的可量化性,便于进行定量分析。

(7) 适用性:指标应适用于不同类型的城市,具有一定的普遍适用性。

\subsection{康养指标}

(1) 生活环境:良好的生活环境有助于减少疾病传播、改善心理状态、提高生活质量,是维持身体健康的重要基础。

(2) 医疗服务:医疗服务能及时预防、诊断和治疗疾病,提升健康管理水平,有效降低死亡率和患病风险,是保障公众健康的关键力量。

(3) 经济水平:经济水平决定了医疗保障、营养摄入和生活质量,高水平经济可提供更完善的健康服务体系,为居民健康提供有力支撑。

(4) 人口结构:人口结构影响健康资源需求与配置,老龄化社会需更多医疗与养老服务,合理的人口结构有助于实现健康服务的可持续发展。

(5) 生活成本:生活成本影响人们获取健康食物、医疗服务与良好居住条件的能力,成本过高可能导致压力增加和健康资源不足,从而危害身心健康。

(6) 教育资源:教育资源提升健康意识与自我管理能力,促使人们养成良好生活习惯,有助于预防疾病、改善心理健康,从根本上促进全民健康水平。

(7) 社区服务:社区服务能提供便捷的医疗、养老、心理支持等服务,增强居民健康管理与应急能力,是提升全民健康水平的重要基础设施。

\subsection{量化康养指标}

\subsubsection{资源密度}
\[
\rho = \frac{R}{S}
\]

其中,$\rho$为资源密度,$R$为资源总量,$S$为区域面积。

\subsubsection{人口密度}
\[
\delta = \frac{P}{S}
\]

其中,$\delta$为人口密度,$P$为区域人口总数,$S$为区域面积。

\subsubsection{资源配置密度/人均资源数量}
\[
E = \frac{\rho}{\delta}\cdot 10^5=\frac{R}{P}\cdot 10^5
\]

其中,$E$为资源配置均衡度,$10^5$校准数量级,$R$为资源总量,$P$为区域人口总数。\textbf{当$E$越大,说明人均资源越丰富。}

\subsubsection{康养资源倾斜度}

\[
W=\frac{R}{C}
\]

其中,$W$为康养资源倾斜度,$R$为资源总量,$C$为年人均收入。

\subsubsection{资源分配系数}

\[
G=\frac{R}{P\cdot C}
\]
其中,$G$为资源分配系数,$R$为资源总量,$P$为区域人口总数,$C$为年人均收入。

\subsubsection{资源配置合理性}
\[
\sigma = \sqrt{\frac{1}{n} \sum_{i=1}^{n}(G_i - \overline{G})^2}
\]
其中,$\overline{G} = \frac{1}{n} \sum_{i=1}^{n}G_i$,是全体$G$的均值,$E_i$为第i个城市的资源配置密度,n为数据数量(个数),$\sigma$为全体E的方差,代表资源配置合理性。\textbf{当$\sigma$越小,说明资源配置越合理。}

\subsubsection{考虑地理因素下的资源配置合理性}

在考虑地理因素的情况下,我们需要引入地理位置对资源配置的影响。地理位置已经由百度地图开放平台获取(图1)。将经纬度转化为平面直角坐标系下的坐标点,使用\textbf{Delaunay三角剖分}\cite{02}方法对上海市的医院进行空间分析。

Delaunay三角剖分能够有效地将空间中的点集划分为一系列三角形(图2),使得每个三角形的顶点都是点集中的点,并且满足某些几何性质,如最大化最小角度等。这种方法在地理信息系统(GIS)中被广泛应用,可以用于分析空间数据的分布特征、计算邻近关系等。

\begin{figure}[H] %H为当前位置,!htb为忽略美学标准,htbp为浮动图形
\centering %图片居中
\includegraphics[width=0.5\textwidth]{images/shanghai_hospital_location.jpg} %插入图片,[]中设置图片大小,{}中是图片文件名
\caption{上海医院坐标的散点图} %最终文档中希望显示的图片标题
\label{Fig.main2} %用于文内引用的标签
\end{figure}

为了得到资源分布合理性$G$的信息,我们选取以下几个指标:

(1) 剖分三角面积$S$:当剖分三角面积$S$越大,说明资源分布越稀疏;当$S$越小,说明资源分布越密集。

\it程序在这一步会获得当前三角形的面积,三角形重心的坐标点,对坐标进行逆地理编码,获得坐标所在区级行政区域,以此该区域的人口密度以及人均收入。
\normalfont

(2) 人口数量$P$:当人口数量$P$越大,说明该区域的人口越密集,资源需求越大。

(3) 人均收入$C$:当人均收入$C$越高,说明该区域的经济水平越高,居民获取资源的能力越强。

\begin{figure}[H] %H为当前位置,!htb为忽略美学标准,htbp为浮动图形
\centering %图片居中
\includegraphics[width=0.5\textwidth]{images/delaunay_triangulation_shanghai.png} %插入图片,[]中设置图片大小,{}中是图片文件名
\caption{上海\textbf{部分}医院坐标的三角剖分示意图} %最终文档中希望显示的图片标题
\label{Fig.main2} %用于文内引用的标签
\end{figure}

于是我们可以定义一个新的资源配置合理性指标$B$,综合考虑以上三个因素:
\[
B = \frac{1}{S\cdot P \cdot C}
\]

其中,$B$为资源配置合理性指标,$S$为剖分三角面积,$P$为人口数量,$C$为人均收入。

由于$P=\rho \cdot S$,$\rho$为人口密度,因此可以将$B$重写为

\[
B = \frac{k}{S^2 \cdot C \cdot \rho}
\]

理想情况下,$B$应趋于一个常数。\textbf{$B$过高或过低都说明资源配置不合理。}

\subsubsection{资源配置均衡度}

为了衡量所有$B$趋近某个常数的程度,我们可以定义资源配置均衡度$I$为:

\[
I = \sqrt{\frac{1}{n} \sum_{i=1}^{n}(B_i - \overline{B})^2}
\]

其中,$\overline{B} = \frac{1}{n} \sum_{i=1}^{n}B_i$为全体$B$的均值,$I$为资源配置均衡度,$n$为数据数量(个数),$B_i$为第$i$个城市的资源配置均衡度指标。

\subsection{简化的康养指数}

在实际应用中,我们往往难以获得大量数据支撑我们的计算,于是我们可以将上述指标进行简化,得到一个综合的康养指数$A$,用于衡量城市的康养资源配置状况。

\subsubsection{康养指数公式}

\[
A = a_1E + a_2W + a_3G + a_4\frac{1}{\sigma}
\]

其中A为康养指数,K为各项指标权重,E为资源配置均衡度,$\sigma$为资源配置合理性,$W$为康养资源倾斜度。这里的简化指标舍去了坐标信息,于是数据收集显得较为容易。

\subsubsection{权重计算}

我们将采用熵值法\cite{03}与专家打分法计算权重。
熵值法是一种客观赋权方法,能够有效消除主观因素对权重的影响。其基本步骤如下:

(1) 数据标准化:将原始数据进行标准化处理,使其符合正态分布。

(2) 计算熵值:根据标准化后的数据,计算各指标的熵值。

(3) 计算权重:根据熵值计算各指标的权重。

(4) 归一化处理:将权重进行归一化处理,使其和为1。

经过计算,得到各指标的权重,得公式如下:

\[
A = 0.166E + 0.030W + 0.671G + 0.133\frac{1}{\sigma}
\]

其中A为康养指数,E为资源配置均衡度,$\sigma$为资源配置合理性,$W$为康养资源倾斜度。


\section{数据计算}

接下来以上海,广州,常州,南京,海口为例,计算资源配置均衡度,并分析其资源分布合理性。
\subsection{资源配置合理性计算}

\begin{table}[h]
  \centering
  \caption{康养城市综合评价指标数据列表}
  \begin{tabular}{c|c|c|c}
    \toprule[2pt]
    城市 &康养指数A&人均寿命/年&排名\\
    \midrule[1pt]
    常州 & 17383.05 & 79.87 & 5 \\
    广州 & 19598.97 & 83.18 & 4 \\
    海口 & 31586.07 & 79.60 & 1 \\
    南京 & 22184.28 & 83.32 & 2 \\
    上海 & 20117.71 & 83.18 & 3 \\
    \bottomrule[2pt]
  \end{tabular}

  \vspace{0.5em}
  {\footnotesize 注:源数据见附录。}
\end{table}

\subsection{地理因素下的资源配置合理性计算}

\begin{figure}[H] %H为当前位置,!htb为忽略美学标准,htbp为浮动图形
\centering %图片居中
\includegraphics[width=0.5\textwidth]{images/hospital.png} %插入图片,[]中设置图片大小,{}中是图片文件名
\caption{上海医院的三角剖分示意图} %最终文档中希望显示的图片标题
\label{Fig.main2} %用于文内引用的标签
\end{figure}

\begin{figure}[H] %H为当前位置,!htb为忽略美学标准,htbp为浮动图形
\centering %图片居中
\includegraphics[width=0.5\textwidth]{images/nursing_home.png} %插入图片,[]中设置图片大小,{}中是图片文件名
\caption{上海养老院的三角剖分示意图} %最终文档中希望显示的图片标题
\label{Fig.main2} %用于文内引用的标签
\end{figure}

\begin{figure}[H] %H为当前位置,!htb为忽略美学标准,htbp为浮动图形
\centering %图片居中
\includegraphics[width=0.5\textwidth]{images/community.png} %插入图片,[]中设置图片大小,{}中是图片文件名
\caption{上海社区的三角剖分示意图} %最终文档中希望显示的图片标题
\label{Fig.main2} %用于文内引用的标签
\end{figure}

\newpage

\section{数据分析与结果}

由3.1结果可以看出,我们的简化康养指数$A$能较好拟合康养城市排名\cite{04},且康养指数较高的城市其人均寿命也相对较高,这表明康养资源配置与居民健康水平之间存在一定的正相关关系。其中,海口市的康养指数最高,排名第一,说明其康养资源配置较为合理,居民健康水平较高;而常州市的康养指数最低,排名第五,说明其康养资源配置相对较差,康养资源的建设有待提高。

由3.2结果可以看出,上海市的康养资源配置均衡度$I$较高,说明其康养资源配置较为均衡。在Delaunay三角剖分的结果中,上海市的医院、养老院和社区的分布较为均匀,说明其康养资源配置较为合理。通过对上海市的医院、养老院和社区进行空间分析,可以看出其康养资源配置存在一定的倾斜度,在市区(普陀,长宁,静安,黄浦等区)红点极为密集,这代表在市区资源丰度极高。但即使市区人口密集,平均收入较高,应有较多康养设施,在数据处理后,B值仍然表征出市区资源有些过于密集,这可能导致资源的浪费。总体上来看,外围呈现深绿色,这表明郊区的资源较少,内圈几乎为白色,这表明市中心外圈的资源平衡是极佳的,在中心能看见少数偏红的区块,但是较少的数量,说明可能在市中心内圈分布了冗余的资源。\textbf{但综合来看,资源分布仍然较为均衡。}

\section{解决方案}

为了解决康养资源分布不均、医养融合深度不够等问题,本文提出以下算法方面的解决方案:

1. \textbf{基于Delaunay三角剖分的动态优化模型}:  
   在现有三角剖分基础上,引入时间维度(如季度人口流动数据),构建动态资源配置算法:  
     利用Python的'scipy.spatial'库实时更新坐标点,计算各行政区三角面积 \( S \)、人口 \( P \)、收入 \( C \) 的动态值,自动生成 \( B \) 值热力图;  
     当某区域 \( B \) 值连续3个月偏离均值±20\%时,触发预警机制,提示规划部门调整资源布局(例如上海浦东陆家嘴区域因商务人口聚集,可临时增设移动医疗站)。  

2. \textbf{多目标优化算法融合}:
   结合熵值法权重与空间分析结果,建立多目标资源配置模型:  
     \[
     \min \sum_{i=1}^{n} I_i + \min \sum_{j=1}^{m} |W_j - \bar{W}| \quad 
     \]
     \[
     \text{s.t.} \quad R_j \geq R_{\text{min}}, \, P_j \leq P_{\text{max}}
     \]  
     其中,\( I_i \) 为区域均衡度,\( W_j \) 为资源倾斜度,通过粒子群优化(PSO)\cite{05}算法\\(见附录Algorithm 1)求解,兼顾均衡性与需求匹配度(如广州天河区需同时满足高收入人群对高端康养服务的需求与人口密集区的基础医疗供给)。  


3. \textbf{智能化资源调度系统}: 
   开发“智能调度平台”,集成以下功能:  
     \textbf{需求预测}:基于历史数据与人口模型(如Leslie\\矩阵),预测各区域未来5年康养需求(如南京江宁区新建住宅区集中,需提前规划社区养老设施);  
     \textbf{路径优化}:利用Dijkstra算法(见附录Algorithm 2)为流动医疗车规划最优路线,确保覆盖\\90\%的三角剖分单元(如上海杨浦区老旧小区密集区域,通过优化路线提升服务效率);  
     \textbf{数据可视化}:通过ECharts动态展示各城市 \( A \) 、\( B \) 指标分布,辅助决策者实时监控资源配置效果。  

4. \textbf{跨区资源协同算法}:
   针对长三角等城市群康养资源流动需求,设计\textbf{区域协同配置模型}:  
   建立跨城市资源池,当某城市
   \onecolumn
   \noindent单一指标异常时(如常州医疗资源密度 \( \rho \) 低于阈值),自动调用周边城市(如南京、上海)的冗余资源进行支援;  
   通过引力模型(见附录Algorithm 3)计算城市间资源协作强度,优先在地理邻近、经济水平相近的城市间(如上海-苏州-嘉兴)建立资源共享机制,降低跨区域调度成本。

\section{结论与讨论}

本文围绕康养城市建设中资源配置不均与空间分布失衡的问题,构建了以康养指数为核心的综合评价模型,并引入Delaunay三角剖分对城市内资源的地理分布进行了深入分析。通过选取常州、广州、南京、海口、上海五个城市为代表,计算康养指数与资源配置合理性指标,验证了模型的有效性与可解释性。

研究表明:

(1) 康养指数越高的城市,其居民人均寿命水平越高,说明康养资源的合理配置对居民健康具有显著促进作用。

(2) 资源配置合理性指标$B$与其均衡性指标$I$能够较好反映城市内部资源是否存在集聚、冗余或缺失现象,具有实际指导意义。

(3) Delaunay三角剖分方法在康养资源空间分析中表现出良好的适应性,能有效识别高密度区与资源盲区,为城市空间规划提供量化依据。

(4) 上海市资源分布虽总体较均衡,但仍存在市区资源过密、郊区资源不足的倾向,提示需在区域层面优化设施布局。

本文模型兼顾了多源数据的综合性与简洁性,具备良好的扩展性和推广价值。未来的研究可以进一步引入动态人口迁移、土地利用变化、政策投入等因素,提升模型的时效性和现实指导力。

综上,本文所提出的康养指数及其评估框架,能够为地方政府进行康养资源布局优化、制定科学发展策略提供有力支撑,也为相关研究提供了可复用的分析思路与方法工具。


\onecolumn
% 附录
\newpage
\section*{附录}

\begin{table}[h]
  \centering
  \caption{康养城市综合评价指标数据列表}
  \begin{adjustbox}{width=\textwidth}
  \begin{tabular}{c|c|c|c|ccccc|c|c}
    \toprule[2pt]
    \multirow{2}{*}{城市} & \multirow{2}{*}{人口数量/万} & \multirow{2}{*}{城市面积/$km^2$}& \multirow{2}{*}{年平均收入/元}& \multicolumn{5}{c|}{公共设施数量} &\multirow{2}{*}{人均寿命/年}&\multirow{2}{*}{康养城市排名\cite{04}}\\
    \cline{5-9}
    & & & & 社区 & 医院 & 养老院 & 公园 & 院校 & &\\
    \midrule[1pt]
    常州 & 538.6 & 4385.00 &59514& 96 & 55 & 89 & 20 & 10 & 79.87 &48\\
    广州 & 1897.8 & 7434.40 &76849& 90 & 73 & 118 & 69 & 35 & 83.18 &2\\
    海口 & 287.3 & 2296.82 &38361& 79 & 63 & 122 & 55 & 12 & 79.60 &1\\
    南京 & 954.7 & 6587.04 &69039& 100 & 85 & 68 & 64 & 80 & 83.32 &11\\
    上海 & 2480.26 & 6340.50 &84034& 58 & 176 & 103 & 112 & 72 & 83.18 &36\\
    \bottomrule[2pt]
  \end{tabular}
  \end{adjustbox}

  \vspace{0.5em}
  {\footnotesize 注:公共设施数量信息来自百度地图。}
\end{table}

\begin{table}[h]
  \centering
  \caption{康养城市综合评价指标数据列表}
  \begin{adjustbox}{width=\textwidth}
  \begin{tabular}{c|c|c|c|ccccc}
    \toprule[2pt]
    \multirow{2}{*}{城市} & \multirow{2}{*}{人口数量/万} & \multirow{2}{*}{城市面积/$km^2$}& \multirow{2}{*}{年平均收入/元}& \multicolumn{5}{c}{公共设施数量}\\
    \cline{5-9}
    & & & & 社区 & 医院 & 养老院 & 公园 & 院校\\
    \midrule[1pt]
    黄浦 & 66.2 & 20.46 & 96448 & 1 & 7 & 3 & 6 & 3 \\
    徐汇 & 111.3 & 54.93 & 90555 & 9 & 12 & 5 & 5 & 11 \\
    长宁 & 69.3 & 38.30 & 92402 & 4 & 5 & 1 & 4 & 5 \\
    静安 & 97.6 & 36.88 & 93547 & 2 & 15 & 5 & 6 & 2 \\
    普陀 & 124.0 & 54.83 & 88916 & 8 & 4 & 3 & 6 & 2 \\
    虹口 & 75.7 & 23.46 & 90959 & 1 & 8 & 4 & 5 & 3 \\
    杨浦 & 124.3 & 60.73 & 90529 & 1 & 13 & 5 & 8 & 7 \\
    闵行 & 265.3 & 372.56 & 82413 & 3 & 8 & 15 & 9 & 4 \\
    宝山 & 223.5 & 365.30 & 79344 & 1 & 5 & 5 & 5 & 5 \\
    嘉定 & 183.4 & 463.55 & 67277 & 1 & 9 & 7 & 10 & 2 \\
    浦东 & 568.1 & 1210.41 & 84089 & 10 & 36 & 16 & 20 & 10 \\
    金山 & 82.3 & 613.00 & 53817 & 1 & 5 & 5 & 2 & 1 \\
    松江 & 191.0 & 604.64 & 66452 & 5 & 4 & 6 & 9 & 3 \\
    青浦 & 127.1 & 668.54 & 59944 & 3 & 5 & 1 & 2 & 0 \\
    奉贤 & 114.1 & 733.38 & 55292 & 2 & 2 & 5 & 3 & 4 \\
    崇明 & 63.8 & 1413.00 & 48237 & 1 & 1 & 11 & 4 & 4 \\
    \bottomrule[2pt]
  \end{tabular}
  \end{adjustbox}

  \vspace{0.5em}
  {\footnotesize 注:公共设施数量信息来自百度地图,其他数据均来自上海市统计局。}
\end{table}

\begin{algorithm}
\caption{粒子群优化算法(PSO)}
\label{alg:pso}
\begin{algorithmic}[1]
\State \textbf{输入}: 粒子群规模 $N$, 维度 $D$, 最大迭代次数 $K_{\max}$
\State \textbf{输出}: 全局最优解 $P_{gbest}$

\For{每个粒子 $i \in \{1,\ldots,N\}$}
    \For{每个维度 $d \in \{1,\ldots,D\}$}
        \State 随机初始化位置 $x_{id} \sim U(x_{min}, x_{max})$
        \State 随机初始化速度 $v_{id} \sim U(-|x_{max}-x_{min}|, |x_{max}-x_{min}|)$
    \EndFor
\EndFor

\State 初始化迭代计数器 $k \gets 1$

\Repeat
    \For{每个粒子 $i \in \{1,\ldots,N\}$}
        \State 计算适应度值 $f(\mathbf{x}_i^k)$
        \If{当前适应度优于个体历史最优 $P_{i,pbest}^k$}
            \State 更新 $P_{i,pbest}^k \gets f(\mathbf{x}_i^k)$
        \EndIf
    \EndFor
    
    \State 选择群体最优 $P_{gbest}^k \gets \arg\min_{i} P_{i,pbest}^k$
    
    \For{每个粒子 $i \in \{1,\ldots,N\}$}
        \For{每个维度 $d \in \{1,\ldots,D\}$}
            \State 更新速度:
            \begin{equation*}
                v_{id}^{k+1} = w v_{id}^k + c_1 r_1 (P_{i,pbest}^k - x_{id}^k) + c_2 r_2 (P_{d,gbest}^k - x_{id}^k)
            \end{equation*}
            \State 更新位置:
            \begin{equation*}
                x_{id}^{k+1} = x_{id}^k + v_{id}^{k+1}
            \end{equation*}
        \EndFor
    \EndFor
    
    \State $k \gets k + 1$
\Until{达到最大迭代次数 $k > K_{\max}$ 或满足误差阈值}

\State \Return 全局最优解 $P_{gbest}$
\end{algorithmic}
\end{algorithm}

\begin{algorithm}
\caption{Dijkstra 最短路径算法}
\label{alg:dijkstra}

\begin{algorithmic}[1]
\Require 
  \Statex 带权图 $G=(V,E)$,边权 $w(e) \geq 0,\ \forall e \in E$
  \Statex 源节点 $s \in V$
\Ensure
  \Statex 源节点 $s$ 到所有节点的最短距离 $dist[\cdot]$
  \Statex 最短路径树的前驱节点 $prev[\cdot]$

\State 初始化:
\For{每个节点 $v \in V$}
  \State $dist[v] \gets \infty$
  \State $prev[v] \gets \text{null}$
\EndFor
\State $dist[s] \gets 0$

\State 创建优先队列 $Q$(按$dist$值排序)
\For{每个节点 $v \in V$}
  \State $Q.\text{insert}(v, dist[v])$
\EndFor

\While{$Q$ 非空}
  \State $u \gets Q.\text{extractMin}()$ \Comment{取出当前距离最小的节点}
  
  \For{每个邻居节点 $v \in \text{Adj}[u]$}
    \State $alt \gets dist[u] + w(u,v)$ \Comment{松弛操作}
    \If{$alt < dist[v]$}
      \State $dist[v] \gets alt$
      \State $prev[v] \gets u$
      \State $Q.\text{decreaseKey}(v, alt)$ \Comment{更新队列优先级}
    \EndIf
  \EndFor
\EndWhile

\State \Return $(dist, prev)$
\end{algorithmic}
\end{algorithm}

\begin{algorithm}
\caption{引力搜索算法(GSA)}
\label{alg:gsa}

\begin{algorithmic}[1]
\State \textbf{输入}: 
\Statex \quad 粒子数 $N$, 维度 $D$, 最大迭代 $T_{\max}$ 
\Statex \quad 初始引力常数 $G_0$, 衰减因子 $\alpha$
\State \textbf{输出}: 最优解 $\mathbf{x}_{best}$

\State 随机初始化粒子位置 $\mathbf{x}_i \in [\mathbf{lb}, \mathbf{ub}],\ i=1,\dots,N$
\State 计算初始质量 $M_i(0)$ 和适应度 $f(\mathbf{x}_i(0))$

\For{$t = 1$ \textbf{to} $T_{\max}$}
    \State 更新引力常数: $G(t) = G_0 e^{-\alpha t/T_{\max}}$
    \State 计算当前最佳/最差适应度:
    \Statex $\quad f_{best}(t) = \min\limits_{j\in\{1,\dots,N\}} f(\mathbf{x}_j(t))$
    \Statex $\quad f_{worst}(t) = \max\limits_{j\in\{1,\dots,N\}} f(\mathbf{x}_j(t))$
    
    \For{每个粒子 $i$}
        \State 计算惯性质量: 
        \Statex $m_i(t) = \frac{f(\mathbf{x}_i(t)) - f_{worst}(t)}{f_{best}(t) - f_{worst}(t)}$
        \State 标准化质量: $M_i(t) = \frac{m_i(t)}{\sum_{j=1}^N m_j(t)}$
    \EndFor
    
    \For{每个粒子 $i$}
        \State 初始化合力 $\mathbf{F}_i = \mathbf{0}$
        \For{每个粒子 $j \neq i$}
            \State 计算欧氏距离: $R_{ij} = \|\mathbf{x}_i(t) - \mathbf{x}_j(t)\|_2$
            \State 计算引力:
            \Statex $\mathbf{F}_{ij} = G(t)\frac{M_i(t)M_j(t)}{R_{ij}+\epsilon}(\mathbf{x}_j(t)-\mathbf{x}_i(t))$
            \State 累加力: $\mathbf{F}_i = \mathbf{F}_i + \text{rand}_j \cdot \mathbf{F}_{ij}$
        \EndFor
        
        \State 计算加速度: $\mathbf{a}_i(t) = \mathbf{F}_i(t)/M_i(t)$
        \State 更新速度: $\mathbf{v}_i(t+1) = \text{rand}_i \cdot \mathbf{v}_i(t) + \mathbf{a}_i(t)$
        \State 更新位置: $\mathbf{x}_i(t+1) = \mathbf{x}_i(t) + \mathbf{v}_i(t+1)$
    \EndFor
\EndFor

\State \Return 全局最优解 $\mathbf{x}_{best}$
\end{algorithmic}
\end{algorithm}
% 参考文献
\newpage
\begin{thebibliography}{99}
        \bibitem{01} 王峰.我省出台意见支持康养产业高质量发展[N].合肥日报,2024-11-13(001).
        \bibitem{02} 百度百科. Delaunay三角剖分[EB/OL]. \url{https://baike.baidu.com/item/%E4%B8%89%E8%A7%92%E5%89%96%E5%88%86?fromModule=lemma_search-box}, 访问时间:2025年5月24日。
        \bibitem{03} 王靖,张金锁.综合评价中确定权重向量的几种方法比较[J].河北工业大学学报,2001,(02):52-57.
        \bibitem{04} 中国康养城市100强名单[EB/OL]. \url{https://www.maigoo.com/news/727956.html}, 访问时间:2025年5月24日。
        \bibitem{05} 刘国海,万亚连,沈跃,等.基于改进粒子群算法的高地隙无人喷雾机对不规则凸田块的全覆盖作业路径规划[J].华南农业大学学报,2025,46(03):390-398.。
        \bibitem{06} 郑自君,袁东升,房鹏,等.攀西地区森林康养指数综合分析[J].气象科技,2021,49(05):815-822.DOI:10.19517/j.1671-6345.20200559.
        \bibitem{07} 百度地图开放平台. \url{https://lbsyun.baidu.com/}, 访问时间:2025年5月24日。
        \bibitem{08} 房红,张旭辉.康养产业:概念界定与理论构建[J].四川轻化工大学学报(社会科学版),2020,35(04):1-20.
        \bibitem{09} 何伟俊,万忠平.基于改进Delaunay三角剖分算法的房屋面群自动综合[J].地理空间信息,2025,23(03):114-117.
        \bibitem{10} 腾讯新闻. 上海和北京的各辖区人均收入排名[EB/OL]. \url{https://news.qq.com/rain/a/20230817A0357X00}, 访问时间:2025年5月22日


\end{thebibliography}

\end{document}
